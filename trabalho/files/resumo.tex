\setlength{\absparsep}{18pt} 
\begin{resumo}[Resumo]
 Segundo a  o resumo deve ressaltar o
 objetivo, o método, os resultados e as conclusões do documento. A ordem e a extensão
 destes itens dependem do tipo de resumo (informativo ou indicativo) e do
 tratamento que cada item recebe no documento original. O resumo deve ser
 precedido da referência do documento, com exceção do resumo inserido no
 próprio documento. (\ldots) As palavras-chave devem figurar logo abaixo do
 resumo, antecedidas da expressão Palavras-chave:, separadas entre si por
 ponto e finalizadas também por ponto. Deve ser redigido na terceira
 pessoa do singular e quanto a sua extensão, o resumo deve ter de 150 a 500
 palavras.

 Palavras-chave: latex. abntex. editoração de texto.
\end{resumo}

\begin{resumo}[Abstract]
 \begin{otherlanguage*}{english}
  É a tradução do resumo para o inglês (Abstract), com a finalidade de facilitar a
  divulgação do trabalho em nível internacional.
 \end{otherlanguage*}

 Keywords: latex. abntex. editoração de texto.
\end{resumo}